\documentclass{article}

\usepackage{english}

\title{A Hill-Climbing Algorithm to Construct Neural Network Automatically -- tested on MNIST}

\author{
  Chung-Chen Chen \\
  Department of Computer Science and Information Engineering\\
  Nation Quemoy University\\
  Kinmen, Taiwan \\
  \texttt{ccc@nqu.edu.tw} \\
}

\begin{document}
\maketitle

\begin{abstract}
Deep learning models are usually constructed by human, based on their intutition.
If we can develop a automatically method to construct neural network model, maybe we can construct some models that's difficult for human to build. 
A method based on Hill Climbing Algorithm is use to build Neural Network model automatically. 
In this paper, we propose a method based on Hill-Climbing Algorithm to construct neural network model automatically. 
\end{abstract}


% keywords can be removed
\keywords{Neural Network\and Deep Learning\and MNIST\and Auto Construct}


\section{Introduction}

\section{Background}

 \cite{kour2014real,kour2014fast} and see \cite{hadash2018estimate}.

The documentation for \verb+natbib+ may be found at
\begin{center}
  \url{http://mirrors.ctan.org/macros/latex/contrib/natbib/natnotes.pdf}
\end{center}
Of note is the command \verb+\citet+, which produces citations
appropriate for use in inline text.  For example,
\begin{verbatim}
   \citet{hasselmo} investigated\dots
\end{verbatim}
produces
\begin{quote}
  Hasselmo, et al.\ (1995) investigated\dots
\end{quote}

\begin{center}
  \url{https://www.ctan.org/pkg/booktabs}
\end{center}

\section{Method}

\subsection{Measure: The height of Hill Climbing Algorithm}


\bibliographystyle{unsrt}  
\bibliography{references}  %%% Remove comment to use the external .bib file (using bibtex).
%%% and comment out the ``thebibliography'' section.


%%% Comment out this section when you \bibliography{references} is enabled.
% \begin{thebibliography}{1}


% \end{thebibliography}


\end{document}
