\documentclass{article}

\usepackage{chinese}
\usepackage{listings}

\lstset{frame=tb,
  language=Python,
}

\title{採用爬山演算法自動建構神經網路模型 -- 以 MNIST 為例}

\author{
  陳鍾誠 \\
  國立金門大學 資訊工程學系\\
  \texttt{ccc@nqu.edu.tw} \\
}

\begin{document}
\maketitle

\begin{abstract}
本論文的開放原始碼專案網址為:\url{ https://github.com/cccresearch/nnModelAuto/ }
\\
\\
以人腦建構神經網路或深度學習模型,通常得依賴研究者的直覺。
但如果能由程式自動建構神經網路的架構,除了不需要依賴人腦的直覺之外,還有可能建構出人腦所難以想出來的模型。
本論文針對手寫數字辨識問題,在 MNIST 資料集上,採用爬山演算法進行了初步的《自動建構神經網路》實驗!
\end{abstract}


% keywords can be removed
\keywords{神經網路\and 深度學習\and MNIST\and 自動建模}


\section{簡介}

\section{背景}

\section{方法}


\renewcommand\refname{參考文獻}

\bibliographystyle{unsrt}  
%\bibliography{references}  %%% Remove comment to use the external .bib file (using bibtex).
%%% and comment out the ``thebibliography'' section.


%%% Comment out this section when you \bibliography{references} is enabled.
\begin{thebibliography}{1}


\end{thebibliography}


\end{document}
